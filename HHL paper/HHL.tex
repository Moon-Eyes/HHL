\documentclass{ctexart}

\usepackage{yzy}

\title{HHL Algorithm 的研究}

\begin{document}
\maketitle
\section*{一 \quad 综述}

线性方程组$A\bm{x}=\bm{b}$的求解是科学与工程中常见的问题. 随着数据规模的增长,处理线性方程组所需的计算时间也随之增加. 传统算法通常需要$\mathcal{O}(N)$的时间来近似求解$N$个未知数的线性方程组,且仅仅是写出解就需要$\mathcal{O}(N)$的时间. 然而许多情况下,我们并不关心完整的解$\bm{x}$,而只需要解的某个函数,例如$\bm{x}^T M \bm{x}$的期望值. 

为了解决这个问题,Harrow、Hassidim 和 Lloyd 在 2009 年提出了一种量子算法\cite{1},通称为
HHL算法. 对于稀疏矩阵$A$($N\times N$维,条件数$\kappa$),最快的经典算法求解$\bm{x}$并估计$\bm{x}^T M \bm{x}$的时间复杂度约为$\mathcal{O}(N^{\kappa})$. 而Harrow 等三人发现HHL算法估计$\bm{x}^T M \bm{x}$的运行时间约为$\log(N)$和$\kappa$的多项式函数;对于小$\kappa$值,则变为$\mathcal{O}(\mathrm{poly}(\log N))$. 由此可见,HHL在通常情况下算法比经典算法快指数级,具有重要应用价值. 

作者三人在论文中首先介绍了$HHL$算法的基本步骤:初始态制备,\textbf{量子相位估计(QPE),受控旋转,逆量子相位估计},测量和最后的期望值评估.其中,$QPE$,受控旋转和逆$QPE$是算法能够有效实现的核心. 

随后文章进行了复杂性分析,指出算法的运行时间主要取决于矩阵$A$的条件数$\kappa$和所需精度$\epsilon$,并给出其时间复杂度为$\mathcal{O}(\log(N)s^2 \kappa^2 / \epsilon)$,其中$s$为矩阵稀疏度.如果$\kappa$和$1/\epsilon$都是$\mathrm{poly}(\log N)$,则算法可以实现指数级加速. 然而,条件数$\kappa$常常随问题规模$N$呈多项式甚至指数增长,这会限制算法的实际优势. 

文章分析,与经典算法比较,HHL算法的优势在于它只需要$\mathcal{O}(\log N)$的量子比特寄存器来存储输入状态$|b\rangle$,而不需要显式地写出矩阵$A$、向量$b$或解$x$的所有分量. 这与经典的蒙特卡洛算法有相似之处,后者通过从概率分布中采样而不是写出所有分量来实现加速.

文章最后还讨论了算法的一些应用或推广,例如广义矩阵求逆,处理病态矩阵等. 

在此基础上,Dervovic  D.等人于2019年发表的著作Quantum linear systems algorithms: a primer\cite{2}的第三章中进一步讨论了$HLL$算法,指出其特别适用于高效求解如下的量子线性系统问题(QLSP):\\

\noindent \textbf{定义(QLSP):}\quad 设$A$是一个$N\times N$的Hermitian矩阵,且$\mathrm{det{(A)}}=1.$ 设$\bm{b}$和$\bm{x}$是$N$维列向量,且满足$\bm{x}=A^{-1}\bm{b}$. 定义$\lceil \log N \rceil$个量子比特上的量子态$|b \rangle $ 为:
\begin{equation}
|b \rangle := \displaystyle \frac{\sum_{i=0} b_i |i\rangle}{\parallel  \sum_{i=0} b_i |i\rangle \parallel_2} \tag{1.1}
\end{equation}
以及$|x \rangle$为:
\begin{equation}
|x \rangle := \displaystyle \frac{\sum_{i=0} x_i |i\rangle}{\parallel  \sum_{i=0} x_i |i\rangle \parallel_2} \tag{1.2}
\end{equation}
其中$b_i$和$x_i$分别是向量$b$和$x$的第$i$个分量. 给定矩阵$A$(通过$Oracle$访问其元素)和状态$|b \rangle$,在$P>1/2$概率下输出一个状态$|\tilde{x} \rangle$,使得$\parallel |\tilde{x} \rangle - |x \rangle \parallel_2 \leq \epsilon$.\\

书中先详细推导了$HHL$算法的实现细节,引入了滤波函数等重要概念;随后进行了详细的误差分析,讨论了相位估计和后选择过程中的错误是如何影响算法性能的. 另外在第四章中,作者还简单讨论了$HHL$算法的改进方案,包括指数级提高的精度对于系统规模的依赖性等.
\section*{二 \quad 论文解析与公式推导}
\subsection*{1.\quad HHL基本原理}

为了实现HHL算法,我们需要3个量子寄存器——寄存器$n_l$用于存储矩阵$A$特征值的二进制表示,寄存器$n_b$用于存储输入状态$|b\rangle$,另有一个辅助寄存器用于在计算过程中执行中间操作. 

首先来做一些数学准备. 我们设$A$的谱分解如下:
\begin{equation}
A=\displaystyle \sum_{j=1}^N\lambda_j | u_j\rangle \langle u_j |,\quad \lambda_j \in \mathbb{R} \tag{2.1}
\end{equation}
其中$\lambda_j$是$A$的特征值,$| u_j\rangle$是对应的特征向量. 由此可以将$A$的逆表示为:
\begin{equation}
A^{-1}=\displaystyle \sum_{j=1}^N \frac{1}{\lambda_j} | u_j\rangle \langle u_j |,\quad \lambda_j \neq 0 \tag{2.2}
\end{equation}
而$|b\rangle$可以按$\{| u_j\rangle \}$展开为:
\begin{equation}
|b\rangle = \displaystyle \sum_{j=1}^N \beta_j | u_j\rangle, \quad b_j \in \mathbb{C}\tag{2.2}
\end{equation}  
由此即可得到$|x\rangle$在$\{| u_j\rangle \}$下的表示:
\begin{equation}
|x\rangle = A^{-1}|b\rangle = \displaystyle \sum_{j=1}^N \frac{\beta_j}{\lambda_j} | u_j\rangle \tag{2.3}
\end{equation}
这样,我们就将求解$|x\rangle$的问题转化为对$|b\rangle$进行演化和测量的问题,最后的输出$|x\rangle$与输入$|b\rangle$在同一个寄存器中.

下面具体来看算法的过程.当然,在之前应先制备好量子态$|b\rangle$. 

首先是$QPE$. 给定一个具有特征态$|u_j\rangle$和对应复特征值$e^{\mathrm{i}\varphi_j t}$的酉算子$U$,QPE技术可以实现以下映射:\cite{3}
\begin{equation}
|0\rangle |u_j\rangle \xrightarrow{QPE} |\widetilde{\varphi}_j\rangle | u_j\rangle, \tag{2.4}
\end{equation}
其中$|\tilde{\varphi}_j\rangle$是$\varphi_j$的二进制表示,精度为$n_l$位. 若取$U=e^{\mathrm{i}A t}$,则由于其具有特征向量$\{| u_j\rangle\}$和对应特征值$\{e^{\mathrm{i}\lambda_j t}\}$,从而QPE操作对于每一个特征态给出:
\begin{equation}
    |0\rangle^{\otimes n_l} |u_j\rangle \xrightarrow{QPE} |\widetilde{\lambda}_j\rangle | u_j\rangle, \tag{2.5}
\end{equation}
从而对于$| b\rangle$有:
\begin{equation}
|0\rangle^{\otimes n_l} |b\rangle \xrightarrow{QPE} \displaystyle \sum_j e^{\mathrm{i}\lambda_j t}|\widetilde{\lambda_j}\rangle_{n_l} | u_j\rangle_{n_b}  \tag{2.6}
\end{equation}


算法的第二步,是以$|\tilde{\lambda}_j\rangle$为控制位,作用受控的$\sigma_y$-旋转.为了实现这一过程,我们引入一个初始态为$|0\rangle$的辅助寄存器作为目标位. 旋转矩阵可以表为:
\begin{equation}
R_y(2\theta)=e^{-\mathrm{i}\theta \sigma_y} = \begin{pmatrix}
\cos\theta & -\sin\theta \\
\sin\theta & \cos\theta
\end{pmatrix},\qquad \theta = \arccos ({C}/{\widetilde{\lambda}_j}) \tag{2.7}
\end{equation}
其中$C$是归一化常数. 受控旋转的结果为:
\begin{equation}
\displaystyle \sum_{j=1}^N \beta_j |\widetilde{\lambda}_j\rangle | u_j\rangle \Bigg( \sqrt{1-\frac{C^2}{\widetilde{\lambda}_j^2}} |0\rangle + \frac{C}{\widetilde{\lambda}_j}|1\rangle \Bigg )\tag{2.8}
\end{equation}
此时我们进行逆QPE操作,得到:
\begin{equation}
|0\rangle^{\otimes n_l} \otimes \displaystyle \sum_{j=1}^N \beta_j | u_j\rangle \Bigg( \sqrt{1-\frac{C^2}{\widetilde{\lambda}_j^2}} |0\rangle + \frac{C}{\widetilde{\lambda}_j}|1\rangle \Bigg )\tag{2.9}
\end{equation}

\clearpage
\begin{figure}[htbp]
  \centering
    \includegraphics[scale=0.25]{1.png}
    \caption*{\textit{图1. \quad HHL算法的基本流程图}:预先制备好输入状态$|b\rangle$;通过先后作用$H^{\otimes n},controlled-U$和$QFT^{\dagger}$实现QPE操作,然后进行受控旋转,再作用第一步的逆操作$QPE^{\dagger}$,最后测量并输出状态$|x \rangle$.}
\end{figure}

最后对辅助位量子比特展开测量,我们得到$| 1\rangle$的概率为:
\begin{equation}
p_1= \displaystyle C^2 \sum_{j=1}^N \frac{ |\beta_j|^2}{\widetilde{\lambda}_j^2} \tag{2.10}
\end{equation}
而测量后的量子态变为:
\begin{equation}
\displaystyle \frac{1}{\sqrt{\sum_{j=1}^{N}}(\beta_j/{\lambda_j})^2}|0\rangle^{\otimes n_l} \Bigg( \displaystyle \sum_{j=1}^N \frac{\beta_j}{{\lambda}_j} | u_j\rangle\Bigg) | 1\rangle  \tag{2.11}
\end{equation}
可见此时三个量子寄存器中不再存在纠缠,其中寄存器$n_b$所处于状态正是我们期待的$|x\rangle$(忽略掉一个归一化系数).

以上在$A$为Hermitian矩阵的条件下说明了HHL算法的有效性. 而当$A$非Hermitian时,只需要构造:
\begin{equation}
    A'\bm{x'} = \bm{b'};\qquad 
A'=\begin{pmatrix}
0 & A \\
A^{\dagger} & 0
\end{pmatrix},\quad \bm{b'}=\begin{pmatrix}
\bm{b} \\
0
\end{pmatrix},\quad \bm{x'}=\begin{pmatrix}
0 \\
\bm{x}
\end{pmatrix} \tag{2.12}
\end{equation}
即可将问题化为我们上文已经讨论过的形式,从而$HHL$算法仍然有效.

最后需要指出的是:HHL算法帮助我们更快地找到了解向量,但代价是它处于量子态,因此我们无法直接获知解的完整信息,而测量并查找其分量需要花费额外的资源. 

\subsection*{2.\quad 算法细节}
我们先来严格描述HHL算法,然后再逐步分析. HHL算法的过程可由如下伪代码表示:\\

\begin{algorithm}[h]
\caption{\quad HHL算法解决QLSP问题}
\label{alg:hhl}
\textbf{输入:} 初始态$\ket{b}$,矩阵$A$(元素通过Oracle访问),参数$t_0=\mathcal{O}(\kappa/\epsilon)$,$T=\widetilde{\mathcal{O}}(\log(N)s^2 t_0)$,$\epsilon$为期望精度.
\begin{algorithmic}[1]
\State 制备输入态$\ket{\Psi_0}^C \otimes \ket{b}^I$,其中$\displaystyle \ket{\Psi_0} = \sqrt{\frac{2}{T}} \ \sum_{\tau=0}^{T-1} \sin \frac{\pi(\tau+\frac{1}{2})}{T} \ket{\tau}^C$;
\State 对输入态应用条件Hamilton演化:$\displaystyle \sum_{\tau=0}^{T-1} \ket{\tau}\!\bra{\tau}^C \otimes e^{iA\tau t_0/T}$;
\State 对寄存器$C$应用量子Fourier变换,新的基态记为$\ket{k}$,$k \in \{0, \ldots, T-1\}$,定义$\tilde{\lambda} := 2\pi k / t_0.$;
\State 附加辅助寄存器$S$,并以$C$为控制执行受控旋转,将$\ket{\tilde{\lambda}}$映射为$\ket{h(\tilde{\lambda})}$;
\State 对寄存器$C$中的“垃圾”信息进行逆QPE操作;
\State 测量寄存器$S$,结果为“well”时返回寄存器$I$,否则返回步骤1;
\State 在$\mathcal{A}_{\text{HHL}}(\ket{b}, A, t_0, T, \epsilon)$上执行$\mathcal{O}(\kappa)$轮\textbf{振幅放大}操作;
\end{algorithmic}
\textbf{输出:} 状态$\ket{\widetilde{x}}$,满足$\|\ket{\widetilde{x}} - \ket{x}\|_2 \leq \epsilon$.
\end{algorithm}
\clearpage
算法的第一步是制备好初始态$\ket{b}^I$,并输入寄存器$I$中. 简单起见,假设存在酉算子$B$和初始状态$\ket{\text{initial}}$可以实现完美精度的制备$B\ket{\text{initial}}^I = \ket{b}^I$,需要的资源为$T_B$个门. 这里$B$是qRAM Oracle,$T_B$是$\ket{b}$维度的多对数函数.

而时钟寄存器$C$则应该被初始化为:
\begin{equation}
\ket{\Psi_0} = \sqrt{\frac{2}{T}}\  \sum_{\tau=0}^{T-1} \sin \frac{\pi(\tau+\frac{1}{2})}{T} \ket{\tau}^C \tag{2.13}
\end{equation}
该状态可在时间$\text{poly}(\log(T/\epsilon_{\Psi}))$内以误差$\epsilon_{\Psi}$制备\cite{4}. 时间 $T=\tilde{\mathcal{O}}(\log(N)s^{2}t_{0})$ 对应于模拟 $e^{iAt}$ 所需的计算步骤数,其中$t_{0}/T$ 是模拟的步长. 

接下来,对输入态作用条件Hamilton演化$\displaystyle \sum_{\tau=0}^{T-1} \ket{\tau}\!\bra{\tau}^C \otimes e^{iA\tau t_0/T}$,得到:
\begin{equation}
\sqrt{\frac{2}{T}}\  \sum_{j=1}^{N} \beta_j \left( \sum_{\tau=0}^{T-1} e^{\mathrm{i}\frac{\lambda_j t_0 \tau}{T}} \sin \frac{\pi(\tau+\frac{1}{2})}{T} \ket{\tau}^C \right) \ket{u_j}^I \tag{2.14}
\end{equation}
此时作量子Fourier变换:将式(2.14)括号内的部分与Fourier 基态$\displaystyle \frac{1}{\sqrt{T}}\sum_{k=0}^{T-1}e^{-{\mathrm{i}2\pi k\tau}/{T}},|k\rangle$ 取内积,即可将即可将寄存器$C$变换为Fourier基$\ket{k}$下的状态:
\begin{equation}
\sum_{j=1}^{N} \beta_j \sum_{k=0}^{T-1} \underbrace{\left( \frac{\sqrt{2}}{T} \sum_{\tau=0}^{T-1} e^{\mathrm{i}\frac{\pi}{T}(\lambda_j t_0 - 2\pi k)} \sin \frac{\pi(\tau+\frac{1}{2})}{T} \right)}_{\alpha_{k|j}} \ket{k}^C \ket{u_j}^I \tag{2.15}
\end{equation}
定义 $\widetilde{\lambda}_{k}:=2\pi k/t_{0}$ 重新标记基础态 $|k\rangle$,得到:
\begin{equation}
\sum_{j=1}^{N}\beta_{j}\sum_{k=0}^{T-1}\alpha_{k|j}\left[\widetilde{\lambda}_{k}\right)^{C}|u_{j}\rangle^{I}\,.
\tag{2.16}
\end{equation}
其中系数 $\alpha_{k|j}$ 定义为:
\begin{equation} 
\alpha_{k|j}=\frac{\sqrt{2}}{T}\sum_{\tau=0}^{T-1}e^{\mathrm{i}\frac{\pi}{T}(\lambda_{j}t_{0}-2\pi k)}\sin\frac{\pi(\tau+\frac{1}{2})}{T} \tag{2.17}
\end{equation}
可以验证,若定义$\delta := \lambda_j t_0 - 2\pi k$,则当$|k - \lambda_j t_0 / 2\pi| \geq 1$时$\alpha_{k|j}$存在上界:\cite{1}
\begin{equation}
\left|\alpha_{k|j}\right|^2 \leq \frac{64\pi^2}{\delta^2} \tag{2.18}
\end{equation}
可见$\left|\alpha_{k|j}\right|$ 仅在 $\lambda_{j}\approx \frac{2\pi k}{t_{0}}$ 时取到较大值.

以上步骤对应于我们前文提及的$QPE$操作,然而其细节有很大的不同. 本质上来说,\textbf{这是因为我们采用了Hamilton模拟技术来实现$controlled-U$的操作,即可以认为HHL中实际运用的是标准QPE技术的一种变体.}

具体来说,标准QPE的执行依赖于QFT和受控$U^{2^k}$操作,需要一系列精确的幂次酉算子($U, U^2, U^4, \ldots$).而在HHL算法中,我们通过Hamilton模拟$e^{\mathrm{i}A\tau t_0/T}$替代了酉算子$U$,并引入时钟寄存器$\ket{\tau}^C$和条件演化操作$\sum_{\tau} \ket{\tau}\!\bra{\tau}^C \otimes e^{iA\tau t_0/T}$,将时间步长离散化. 此时,我们依赖几何序列求和来分析相位误差,而非直接测量寄存器$C$的二进制位.

因此,$HHL$中“QPE”操作的误差来源于步长$t_0/T$的离散取值;在后续过程中,滤波函数对特征值的截断(见后文)也会对特征值的估计产生影响. 

不过需要注意的是,HHL中的“QPE”操作和标准QPE操作\textbf{在数学上仍然是等价的}. 两者都是基于QFT的原理,制备不同寄存器之间的量子纠缠态,并通过受控门提取相位信息. 事实上,我们当然可以直接采用标准QPE的方式来实现HHL算法作为简化,但这可能会导致额外的资源消耗. 
\clearpage
算法的下一步,是引入一个辅助寄存器 $S$用于执行受控反转.

这里需要重点考虑的是算法的\textbf{数值稳定性}. 在HHL算法的运用过程中,我们希望只翻转矩阵条件“良好”的部分,即特征值位于某个范围内的部分,该范围相对于 $1/\kappa$ 较大(否则 $A^{-1}$ 的返回值将与真实值偏离到不可接受的程度).

为了实现这一点,我们引入\textbf{滤波器函数} $f(\lambda)$ 和 $g(\lambda)$,它们的作用是仅在 $A$ 的良好条件子空间(即由特征值 $\lambda \geq 1/\kappa$ 对应的特征向量张成的子空间)上反转 $A$,确保 $\lambda$ 的微小误差不会在 $A^{-1}$ 中引入大的误差. 

在HHL算法中,受控旋转由形式为 $\theta=\arccos x$ 的角度控制,其中 $x$ 是滤波器函数对 $\lambda$ 的输出. $\arccos(\cdot)$ 的任何参数都需要在区间 $[-1,1]$ 内,因此滤波器函数的值域必须是 $[-1,1]$。滤波器函数的定义域是 $[\lambda_{min},\lambda_{max}]$. 我们要求在良好条件子空间中,滤波器函数与 $1/\lambda$ 成比例,以实现特征值反转;我们还关注中间特征值,即 $1/\kappa^{\prime}\leq\lambda\leq 1/\kappa$,在这些情况下需要采取插值行为以提高数值稳定性. 满足所有这些要求的滤波器函数选择之一为:

\[f(\lambda)=\begin{cases}\frac{1}{2\kappa\lambda},&\lambda\geq 1/\kappa;\\\frac{1}{2}\sin\left(\frac{\pi}{2}\cdot\frac{\lambda-\frac{1}{\kappa}}{\frac{1}{\kappa}-\frac{1}{\kappa^{\prime}}}\right),&\frac{1}{\kappa}>\lambda>\frac{1}{\kappa^{\prime}};\\0,&\frac{1}{\kappa^{\prime}}>\lambda;\end{cases}\qquad\quad g(\lambda)=\begin{cases}0,&\lambda\geq 1/\kappa;\\\frac{1}{2}\cos\left(\frac{\pi}{2}\cdot\frac{\lambda-\frac{1}{\kappa}}{\frac{1}{\kappa}-\frac{1}{\kappa^{\prime}}}\right),&\frac{1}{\kappa}>\lambda>\frac{1}{\kappa^{\prime}};\\\frac{1}{2},&\frac{1}{\kappa^{\prime}}>\lambda.\end{cases}
\tag{2.19}\]
注意这一步在算法中引入了 $\kappa$ 依赖性.

那么,在受控旋转之后,寄存器 $S$ 的状态将变为:

\[|h(\widetilde{\lambda}_{k})\rangle^{S}:=\sqrt{1-f(\widetilde{\lambda}_{k})^{2}-g(\widetilde{\lambda}_{k})^{2}}\left|\text{nothing}\right\rangle^{S}+f(\widetilde{\lambda}_{k})\left|\text{well}\right\rangle^{S}+g(\widetilde{\lambda}_{k})\left|\text{ill}\right\rangle^{S}.
\tag{2.20}\]

其中标注“nothing”表示未进行反转,“well”表示已反转,“ill”表示 $|b\rangle$ 处于 $A$ 的不良条件子空间的部分。

应用滤波器函数后,我们执行逆QPE过程,同时清除计算过程中的“垃圾”量子位. 到这里为止,我们就完成了一般HHL算法的核心步骤.

然而,就像式(2.10)展现的那样,如果算法到此为止,我们将不能以1的概率在寄存器$C$上测得$|1\rangle$,从而会导致很多冗余操作无法获得$|x\rangle$的有效估计.为此,我们需要采取\textbf{振幅放大}的技术.\cite{3}现在,
 若将到目前为止描述的过程记为 $U_{\text{invert}}$,则将 $U_{\text{invert}}$ 应用于 $|b\rangle$ 并测量 $S$ 得到结果“well”时,返回状态 $|\tilde{x}\rangle$,成功的概率将为 $\tilde{p}=\mathcal{O}\left(1/\kappa^{2}\right)$. 运用振幅放大的原理,通过 $\mathcal{O}\left(1/\sqrt{\tilde{p}}\right)=\mathcal{O}\left(\kappa\right)$ 次重复,我们即可以实现任意高的成功概率. 

之前提到,制备状态 $|b\rangle$ 需要 $\widetilde{O}(T_{B})$,运行量子模拟需要 $\widetilde{O}(t_{0}s^{2}\log N)$。因此,算法的总运行时间为 $\widetilde{O}(\kappa(T_{B}+t_{0}s^{2}\log N))$,其中因子 $\kappa$ 是由于幅度放大。由于 $t_{0}=\mathcal{O}\left(\kappa/\epsilon\right)$,运行时间可以表示为 $\widetilde{O}(\kappa T_{B}+\kappa^{2}s^{2}\log(N)/\epsilon)$.
\section*{三 \quad 个人分析: 初始态制备的优化 }
关于初始态制备的假设,HHL算法要求输入态$|b\rangle$能高效制备,然而这在通用场景下并非易事. 

一个创新思路是将状态制备与矩阵逆的计算融合为单一量子过程。例如,若$|b\rangle$由某Hamilton量$H'$的基态制备,则可设计联合量子线路同时对角化$H'$和$A$,通过量子虚时间演化模拟$e^{-\tau H_{\text{joint}}}|ψ_{\text{init}}⟩$,其稳态即有可能包含$|A^{-1}b⟩$的分量,从而直接输出$|A^{-1}b⟩$. 这种方法需解决多体Hamilton量模拟的复杂度问题,但有望减少预处理步骤.


为了进一步探讨状态制备与矩阵求逆的融合策略,我们首先明确问题设定:给定一个需要求解的线性系统 \( A\mathbf{x} = \mathbf{b} \),其中 \( \mathbf{b} \) 是某哈密顿量 \( H' \) 的基态(即 \( |b\rangle = |\psi_0'\rangle \),满足 \( H'|\psi_0'\rangle = \lambda_0'|\psi_0'\rangle \))。传统HHL算法需独立完成 \( |b\rangle \) 的制备和 \( A^{-1} \) 的作用,而我们的目标是设计一个联合量子过程,直接输出 \( |x\rangle \propto A^{-1}|b\rangle \). 下面我们通过尝试来逐步实现这一目标.
\subsection*{1.\quad 尝试构造联合Hamilton量}
首先,尝试定义联合哈密顿量:
\[
H_{\text{joint}} = H' \otimes I + I \otimes A \tag{3.1}
\]
其作用在复合系统 \( \mathcal{H}_{\text{total}} = \mathcal{H}_{H'} \otimes \mathcal{H}_A \) 上. 假设 \( H' \) 和 \( A \) 分别具有谱分解:
\[
H' = \sum_i \lambda_i' |\psi_i'\rangle\langle\psi_i'|, \quad A = \sum_j \lambda_j |u_j\rangle\langle u_j| \tag{3.2}
\]
则 \( H_{\text{joint}} \) 的本征态为直积态 \( |\psi_i'\rangle \otimes |u_j\rangle \),对应本征值 \( \lambda_i' + \lambda_j \). 特别地,基态为 \( |\psi_0'\rangle \otimes |u_{\text{min}}\rangle \),其中 \( |u_{\text{min}}\rangle \) 是 \( A \) 的最小本征值态.

现在,我们进行量子虚时间演化(Quantum Imaginary Time Evolution, QITE)操作,通过非酉算子 \( e^{-\tau H} \) 将初始态 \( |\psi_{\text{init}}\rangle \) 投影至基态. 对于联合系统而言,我们可以模拟:
\[
|\psi(\tau)\rangle = \frac{e^{-\tau H_{\text{joint}}}|\psi_{\text{init}}\rangle}{\|e^{-\tau H_{\text{joint}}}|\psi_{\text{init}}\rangle\|_2} \tag{3.3}
\]
若初始态为 \( |\psi_{\text{init}}\rangle = |\psi_0'\rangle \otimes |\phi\rangle \)(其中 \( |\phi\rangle \) 任意),当 \( \tau \to \infty \) 时,稳态将趋近于 \( |\psi_0'\rangle \otimes |u_{\text{min}}\rangle \). 然而,我们需要的是 \( A^{-1}|b\rangle \),而非 \( A \) 的基础态. 为此,必须调整哈密顿量构造. 

考虑定义修正的联合哈密顿量:
\[
\tilde{H}_{\text{joint}} = H' \otimes I + I \otimes A^{-1} \tag{3.4}
\]
但直接实现 \( A^{-1} \) 的哈密顿量似乎也不太现实. 这里,我们不妨利用线性组合式哈密顿量模拟(Hamiltonian Simulation via Linear Combination):
\[
\tilde{H}_{\text{joint}} = H' \otimes I + \gamma (I \otimes A)^{-1}\tag{3.5}
\]
其中 \( \gamma \) 为调节参数. 此时通过Trotter-Suzuki分解\cite{4},我们即可近似实现:
\[
e^{-i\tilde{H}_{\text{joint}}t} \approx \left(e^{-iH' \otimes I \cdot t/n} e^{-i\gamma (I \otimes A)^{-1} \cdot t/n}\right)^n \tag{3.6}
\]
但 \( (I \otimes A)^{-1} \) 的模拟仍需通过HHL的子过程,从而将导致循环依赖的出现.

\subsection*{2.\quad 可行的优化方案:直接对角化与量子奇异值变换(QSV)}
受Dervovic  D. 等人的书第四章内容\cite{2}启发,我们可以利用量子奇异值变换(QSV)同时对 \( H' \) 和 \( A \) 进行对角化!设 \( A \) 的奇异值分解为 \( A = \sum_j \sigma_j |u_j\rangle\langle v_j| \),构造块编码(Block Encoding):
\[
U_A = \begin{pmatrix}
A/\alpha & \cdot \\
\cdot & \cdot
\end{pmatrix} \tag{3.7}
\]
其中 \( \alpha \geq \|A\| \). 类似地,对 \( H' \) 构造 \( U_{H'} \)。通过QSV的多项式变换,设计 \( P(A) \approx A^{-1} \) 和 \( P(H') \approx |\psi_0'\rangle\langle\psi_0'| \),使得联合操作:
\[
U_{\text{joint}} = U_{H'}^{\dagger} U_A^{-1} \tag{3.8}
\]
作用于初始态 \( |0\rangle|\phi\rangle \) 时,输出态为 \( |0\rangle A^{-1}|b\rangle + |\text{err}\rangle \).具体步骤为:

\noindent \textbf{1.制备 \( |b\rangle \) 的近似态}:
    利用QSV多项式 \( P(H') \) 逼近投影算子 \( |\psi_0'\rangle\langle\psi_0'| \),使得:
    \[
    P(H')|\phi\rangle \approx |\psi_0'\rangle\langle\psi_0'|\phi\rangle \propto |b\rangle \tag{3.9}
    \]

   \noindent  \textbf{2. 应用 \( A^{-1} \) 的近似}:  
    设计另一多项式 \( Q(A) \approx A^{-1} \),满足 \( Q(\sigma_j) \approx \sigma_j^{-1} \). 通过QSV实现:
    \[
    Q(A)|b\rangle \approx \sum_j \sigma_j^{-1} \beta_j |u_j\rangle \propto |x\rangle \tag{3.10}
    \]

那么,此时的总误差将由以下两部分组成:
    \[
    \||\tilde{x}\rangle - |x\rangle\|_2 \leq \|P(H') - |\psi_0'\rangle\langle\psi_0'|\| \cdot \|Q(A)\| + \|Q(A) - A^{-1}\| \cdot \||b\rangle\|_2 \tag{3.11}
    \]
    若 \( P(H') \) 和 \( Q(A) \) 分别为 \( \epsilon_1 \) 和 \( \epsilon_2 \) 近似,则总误差为 \( O(\epsilon_1 + \epsilon_2) \).

\subsection*{3.\quad 复杂度讨论}
传统HHL状态制备复杂度 \( T_B \),矩阵求逆复杂度 \( O(\kappa^2 s^2 \log N/\epsilon) \). 而联合QSV方法满足:若 \( H' \) 和 \( A \) 的块编码复杂度为 \( O(s') \),则总复杂度取决于多项式次数 \( d \)(与 \( \kappa \)、\( \epsilon \) 相关). 对于条件数 \( \kappa \),需 \( d = O(\kappa \log(1/\epsilon)) \),整体复杂度可能优于分步处理.

综上,通过联合哈密顿量对角化或量子奇异值变换,有望将状态制备与矩阵求逆融合为单一量子过程,但需权衡近似精度与资源开销. 这一方向为减少HHL预处理步骤提供了新思路,值得进一步研究. 

\clearpage
\section*{四 \quad 数值计算}

作为一个例子,我们来求解$\mathrm{https://qiskit.org/textbook/ch-applications/hhl\_tutorial.html}$ 
\cite{4}中给出的简单问题:
\begin{equation}
A\bm{x} =\bm{ b};\qquad A=\begin{pmatrix}
1 & -{1}/{3} \\
-{1}/{3} & 1
\end{pmatrix},\quad \bm{b}=\begin{pmatrix}
1 \\
0
\end{pmatrix} \tag{4.1}
\end{equation}
\subsection*{1.\quad 理论推导}
先来从理论上预测算法的运行结果. 不难求出$A$的特征值和相应特征矢量为:
\begin{equation}
\lambda_1 = \frac{2}{3},\quad \lambda_2 = \frac{4}{3};\qquad |u_1\rangle = \frac{1}{\sqrt{2}}\begin{pmatrix}
1\\
1
\end{pmatrix},\quad |u_2\rangle =\displaystyle \frac{1}{\sqrt{2}} \begin{pmatrix}
1\\
-1
\end{pmatrix} \tag{4.2}
\end{equation}
从而$| b\rangle$ 在该基下的表示是:
\begin{equation}
|b\rangle = |0\rangle = \frac{1}{\sqrt{2}}\left( |u_1\rangle + |u_2\rangle \right) \tag{4.3}
\end{equation}
不过需要注意的是:\textbf{实现HHL算法不必预先知道这些信息}.

我们采用$n_b=1$来寄存$| b\rangle$,$n_l=2$来寄存$A$特征值的二进制表示. 若选取时间步长$\displaystyle t=2\pi\cdot \frac{3}{8}$,则QPE给出的估计为:
\begin{equation}
\frac{\lambda_1 t}{2\pi}=\frac{1}{4},\quad \frac{\lambda_2 t}{2\pi}=\frac{1}{2}\quad \Longrightarrow \quad  |\widetilde{\lambda}_1\rangle = |01\rangle_{n_l},\quad |\widetilde{\lambda}_2\rangle = |10\rangle_{n_l} \tag{4.4}
\end{equation}
从而算法的第一步给出:
\begin{equation}
|0\rangle^{\otimes n_l} |b\rangle \  \xrightarrow{QPE} \ \frac{1}{\sqrt{2}} |01\rangle  |u_1\rangle + \frac{1}{\sqrt{2}}|10\rangle|u_2\rangle \tag{4.5}
\end{equation}
随后,我们进行$C=\displaystyle \frac{1}{8}$的受控旋转,得到:
\begin{equation}
\frac{1}{\sqrt{2}}|01\rangle|u_{1}\rangle\left( \sqrt { 1 - \frac { 1  } {4 } } |0\rangle + \frac { 1 } { 2 } |1\rangle \right) + \frac{1}{\sqrt{2}}|10\rangle|u_{2}\rangle\left( \sqrt { 1 - \frac { 1  } {16 } } |0\rangle + \frac { 1 } { 4 } |1\rangle \right) \tag{4.6}
\end{equation}
最后再作用$QPE^{\dagger}$,得到:
\begin{equation}
|00\rangle \otimes \left[ \frac{1}{\sqrt{2}}|u_1\rangle\left( \sqrt { \frac{3}{4}} |0\rangle + \frac { 1 } { 2 } |1\rangle \right) + \frac{1}{\sqrt{2}}|u_2\rangle\left( \sqrt {\frac{15}{16} } |0\rangle + \frac { 1 } { 4 } |1\rangle \right) \right] \tag{4.7}
\end{equation}
对辅助量子比特进行测量,得到的态为$|1\rangle$的概率为:
\begin{equation}
p_1 =\displaystyle \left(\frac{1}{2\sqrt{2}}\right)^2+\left(\frac{1}{4\sqrt{2}}\right)^2=\frac{5}{32} \tag{4.8}
\end{equation}
此时测量后的量子态变为了:
\begin{equation}
   |00\rangle\otimes \frac{\frac{1}{2\sqrt{2}}|u_{1}\rangle | +\frac{1}{4\sqrt{2}}|u_{2}\rangle}{\sqrt{5/32}} \otimes |1\rangle \tag{4.9}
\end{equation}
很容易用初等方法求得原方程的解为$\displaystyle |x\rangle =(9/8,\ 3/8)^T= \frac{3}{4\sqrt{2}}(2|u_1\rangle + |u_2\rangle),\quad \parallel x \parallel=3\sqrt{\frac{5}{32}} $,从而测量后得到的态正可以写为:
\begin{equation}
|00\rangle_{n_l}\otimes \frac{|x\rangle}{ \Vert x \Vert }\otimes |1\rangle_{n_b} \tag{4.10}
\end{equation}


\clearpage
\subsection*{2.\quad 代码实现}
\noindent 我们来编程实现上述算法. 作为参照,首先采用经典方法来求解该方程组:
\begin{pythoncode}
import numpy as np

matrix = np.array([[1, -1/3], [-1/3, 1]])
vector = np.array([1, 0])
classical_solution = NumPyLinearSolver().solve(matrix,
                                               vector/np.linalg.norm(vector))
print('classical state:', classical_solution.state)

\end{pythoncode}
结果如下:
\begin{figure}[htbp]
  \centering
    \includegraphics[scale=0.5]{4.1.png}
    \caption*{图2. \quad 经典算法的求解结果.}
\end{figure}

\noindent 再来看$HHL$算法的结果. 这里我们将使用$\mathrm{https://github.com/anedumla/quantum\_linear\_solvers}$提供的包$linear\_solvers$,它针对$HLL$算法的实现定义了一系列模块. 这里$HHL$的算法有两种执行模式,一为默认的求解普通矩阵方程的方法,一为利用TridiagonalToeplitz类求解三对角矩阵方程的方法(我们选取的示例恰好满足这个要求). 我们分别画出两种方法的线路图:
\begin{pythoncode}
from linear_solvers import NumPyLinearSolver, HHL
from linear_solvers.matrices.tridiagonal_toeplitz import TridiagonalToeplitz
tridi_matrix = TridiagonalToeplitz(1, 1, -1 / 3)
tridi_solution = HHL().solve(tridi_matrix, vector)

naive_hhl_solution = HHL().solve(matrix, vector) 
print('naive state:')
print(naive_hhl_solution.state)
print('tridiagonal state:')
print(tridi_solution.state)
\end{pythoncode}
结果如下:
\begin{figure}[htbp]
 \centering
  \subfloat[$naive\  state$]
  {\includegraphics[width=0.427\textwidth]{4.2.png}}
  \qquad
  \subfloat[$tridiagonal \ state$]
  {\includegraphics[width=0.45\textwidth]{4.3.png}}
  \caption*{图$3.\quad$算法的线路图}
 \end{figure}


\noindent 接下来计算解向量的Euclidean范数,并逐个分量地来比较解向量(已做归一化):
\begin{pythoncode}
print('classical Euclidean norm:', classical_solution.euclidean_norm)
print('naive Euclidean norm:', naive_hhl_solution.euclidean_norm)
print('tridiagonal Euclidean norm:', tridi_solution.euclidean_norm)
\end{pythoncode}
\clearpage
\begin{pythoncode}
from qiskit.quantum_info import Statevector

naive_sv = Statevector(naive_hhl_solution.state).data
tridi_sv = Statevector(tridi_solution.state).data
naive_full_vector = np.array([naive_sv[16], naive_sv[17]])
tridi_full_vector = np.array([tridi_sv[16], tridi_sv[17]])

def get_solution_vector(solution):
    """Extracts and normalizes simulated state vector
    from LinearSolverResult."""
    solution_vector = Statevector(solution.state).data[16:18].real
    norm = solution.euclidean_norm
    return norm * solution_vector / np.linalg.norm(solution_vector)

print('full naive solution vector:', get_solution_vector(naive_hhl_solution))
print('full tridi solution vector:', get_solution_vector(tridi_solution))
print('classical state:', classical_solution.state)
\end{pythoncode}
\noindent 结果如下:
\begin{figure}[htbp]
  \centering
    \includegraphics[scale=0.45]{4.4.png}
    \caption*{图4. \quad 解向量范数的结果对比.}
\end{figure}
\begin{figure}[htbp]
  \centering
    \includegraphics[scale=0.45]{4.5.png}
    \caption*{图5. \quad 解向量各分量的对比.}
\end{figure}

默认方法得到解是精确的,这并不奇怪;然而TridiagonalToeplitz仅能给出 $2\times 2$矩阵系统的精确解,对于更大的系统,它给出的将是近似值. 

随着系统规模的增加,默认精确算法所需的量子比特数呈指数增长,而三对角矩阵方法所需的比特数仅仅呈多项式增长. 作为验证,我们来比较不同系统规模下两种方法的量子电路层数:\\
\begin{pythoncode}
from scipy.sparse import diags
from qiskit import transpile

MAX_QUBITS = 4
a = 1
b = -1/3
i = 1
# calculate the circuit depths for different number of qubits to compare the use
# of resources (WARNING: This will take a while to execute)
naive_depths = []
tridi_depths = []
for n_qubits in range(1, MAX_QUBITS+1):
    matrix = diags([b, a, b],
                   [-1, 0, 1],
                   shape=(2**n_qubits, 2**n_qubits)).toarray()
    vector = np.array([1] + [0]*(2**n_qubits -1))

    naive_hhl_solution = HHL().solve(matrix, vector)
    tridi_matrix = TridiagonalToeplitz(n_qubits, a, b)
    tridi_solution = HHL().solve(tridi_matrix, vector)

    naive_qc = transpile(naive_hhl_solution.state,
                         basis_gates=['id', 'rz', 'sx', 'x', 'cx'])
    tridi_qc = transpile(tridi_solution.state,
                         basis_gates=['id', 'rz', 'sx', 'x', 'cx'])

    naive_depths.append(naive_qc.depth())
    tridi_depths.append(tridi_qc.depth())
    i +=1
\end{pythoncode}
\begin{pythoncode}
sizes = [f"{2**n_qubits}×{2**n_qubits}"
        for n_qubits in range(1, MAX_QUBITS+1)]
columns = ['size of the system',
           'quantum_solution depth',
           'tridi_solution depth']
data = np.array([sizes, naive_depths, tridi_depths])
ROW_FORMAT ="{:>23}" * (len(columns) + 2)
for team, row in zip(columns, data):
    print(ROW_FORMAT.format(team, *row))

print('excess:',
      [naive_depths[i] - tridi_depths[i] for i in range(0, len(naive_depths))])
\end{pythoncode}
\noindent 结果如下:
\begin{figure}[htbp]
  \centering
    \includegraphics[scale=0.3]{4.6.png}
    \caption*{图6. \quad 两种方法的量子线路层数随系统规模变化的对比.}
\end{figure}



\noindent 回到我们求解方程的主题. 还可以尝试计算解向量的平均分量:
\begin{pythoncode}
from linear_solvers.observables import AbsoluteAverage, MatrixFunctional

NUM_QUBITS = 1
MATRIX_SIZE = 2 ** NUM_QUBITS
# entries of the tridiagonal Toeplitz symmetric matrix
a = 1
b = -1/3

matrix = diags([b, a, b],
               [-1, 0, 1],
               shape=(MATRIX_SIZE, MATRIX_SIZE)).toarray()
vector = np.array([1] + [0]*(MATRIX_SIZE - 1))
tridi_matrix = TridiagonalToeplitz(1, a, b)

average_solution = HHL().solve(tridi_matrix,
                               vector,
                               AbsoluteAverage())
classical_average = NumPyLinearSolver(
                        ).solve(matrix,
                                vector / np.linalg.norm(vector),
                                AbsoluteAverage())

print('quantum average:', average_solution.observable)
print('classical average:', classical_average.observable)
print('quantum circuit results:', average_solution.circuit_results)
\end{pythoncode}
结果如下:
\begin{figure}[htbp]
  \centering
    \includegraphics[scale=0.45]{4.7.png}
    \caption*{图7. \quad 解向量平均分量的对比.}
\end{figure}

还可以提取解向量的更多特征供HHL算法和经典算法展开对比,由于空间所限就不一一详述了.

\subsection*{3.\quad 拓展讨论}
本人独立地撰写了一套基于IBM Qiskit的程序用于实现$HHL$算法,没有额外调用其他平台的包. 所有函数的定义都直观地呈现在了代码中,可以让我们清晰地看到算法的逻辑和步骤. 

这里我们不再使用github上的例子,而是重新考虑一个简单的问题:
\begin{equation}
A\bm{x} =\bm{ b};\qquad A=\begin{pmatrix}
1 & 0 \\
0 & -1 
\end{pmatrix},\quad \bm{b}=\begin{pmatrix}
0.8 \\
0.6
\end{pmatrix} \tag{4.11}
\end{equation}
与github上例子不同的是,这里的相位估计无法做到绝对精确,因此运行结果的准确性高度依赖于参数$C$和$n_l$的选取. 我们通过在不同参数区域妥善选取参数值调试代码,已成功验证$HHL$算法的准确性和有效性,并指出默认算法中进行精确的受控旋转的确需要指数级资源. 作为一个例子,我们选取参数为$C=0.01$,$dt=\pi/8$,$n_{l}=10$来执行计算过程:
\begin{pythoncode}
import numpy as np
from scipy.linalg import expm
from qiskit import QuantumCircuit, transpile
from qiskit.circuit.library import RYGate
from qiskit.providers.basic_provider import BasicSimulator
from qiskit.quantum_info import Operator
import matplotlib.pyplot as plt
import time

def qft(n_qubits):
    circuit = QuantumCircuit(n_qubits)
    circuit.h(range(n_qubits))                # 初始化钟表的指针
    for i in reversed(range(n_qubits)):
        circuit.barrier(range(n_qubits))
        for j in reversed(range(i)):
            circuit.crx(np.pi / 2 ** (i - j),
                        j,                    # 控制位bit
                        i)                    # 目标位bit
    return circuit

def qft_dagger(n_qubits):
    return qft(n_qubits).inverse()

def controlled_U(n_qubits, A, dt):
    circuit = QuantumCircuit(n_qubits + 1)
    U = np.matrix(expm(1j * A * dt)) 
    for i in range(n_qubits):
        sub_circuit = QuantumCircuit(1)  
        sub_circuit.name = '$U^{2^' + str(i) + '}$'
        sub_circuit.append(Operator(U ** (2 ** i)), [0])         
        sub_gate = sub_circuit.to_gate().control()             
        circuit.append(sub_gate, [n_qubits - i - 1, n_qubits])
    return circuit


def controlled_R(n_qubits, C, dt):
    circuit = QuantumCircuit(n_qubits + 1)
    for k in range(1, 2 ** n_qubits):
        for i in range(n_qubits):  
            if (k & (1 << i)) == 0:
                circuit.x(1 + i)
        phi = k / (2 ** n_qubits)
        if k > 2 ** (n_qubits - 1):
            phi -= 1.0  
        l = 2 * np.pi / dt * phi
        circuit.append(RYGate(2 * np.arcsin(C / l)).control(n_qubits),
                       [(i + 1) % (n_qubits + 1) for i in range(n_qubits + 1)])
        for i in range(n_qubits): 
            if (k & (1 << i)) == 0:
                circuit.x(1 + i)
        circuit.barrier(range(n_qubits + 1))
    return circuit

if __name__ == "__main__":
    import scipy.linalg

    dt = np.pi / 8    # time step in e^iHdt
    C = 0.01          # constant in controlled-R
    n_qubits = 10      
    A = np.matrix([
        [1,  0],
        [0, -1]
    ])

    # 初始化
    qc = QuantumCircuit(1 + n_qubits + 1, 2)
    qc.ry(2 * np.arcsin(0.6), n_qubits + 1)

    # 量子相位估计
    qc.barrier(range(1 + n_qubits + 1))
    qc.h(range(1, n_qubits + 1))
    qc.append(controlled_U(n_qubits, A, dt), range(1, 1 + n_qubits + 1))
    qc.append(qft_dagger(n_qubits),range(1, n_qubits + 1))
    qc.barrier(range(1 + n_qubits + 1))

    # 受控旋转
    qc.append(controlled_R(n_qubits, C, dt), range(n_qubits + 1))

    # 逆量子相位估计
    qc.barrier(range(1 + n_qubits + 1))
    qc.append(qft(n_qubits),range(1, n_qubits + 1))
    qc.append(controlled_U(n_qubits, A, dt).inverse(), range(1, 1 + n_qubits + 1))
    qc.h(range(1, n_qubits + 1))
    qc.barrier(range(1 + n_qubits + 1))

    # 测量
    qc.measure(0, 0)
    qc.measure(1 + n_qubits, 1)
    
    # 绘制电路图
    decomposed_qc = qc.decompose()
    fig = decomposed_qc.draw('mpl', fold=-1, scale=0.8)
    plt.title(f"HHL Quantum Circuit ({n_qubits} Clock Qubits)", fontsize=14)
    plt.savefig('HHL_detailed.png', bbox_inches='tight', dpi=300)
    print("电路图已保存为 HHL_detailed.png")
    
    # 运行结果
    print("运行量子模拟:")
    start_time = time.time()
    backend = BasicSimulator()
    t_qc = transpile(qc, backend)
    result = backend.run(t_qc, shots=2 ** 21).result()

    end_time = time.time()  
    elapsed_time = end_time - start_time 
    counts = result.get_counts(t_qc)
    print(counts)
    count_01 = counts.get('01', 0)
    count_11 = counts.get('11', 0)
    total = count_01 + count_11

if total > 0:
    Prob_01 = count_01 / total
    Prob_11 = count_11 / total
    amplitude_01 = np.sqrt(Prob_01)
    amplitude_11 = np.sqrt(Prob_11)
    print("Prob_01:", Prob_01)
    print("Prob_11:", Prob_11)
    print(f"x=({amplitude_01}, {amplitude_11})")
else:
    print("未测量到有效解!")

print(f"运行时间:{elapsed_time:.2f}秒")
\end{pythoncode}
\clearpage
\noindent 算法具体到门的示意图如下(简洁起见,这里展示$n_l=2$情形):

\begin{figure}[htbp]
  \centering
    \includegraphics[scale=0.35]{HHL_detailed.png}
    \caption*{图8. \quad HHL算法的量子线路图.}
\end{figure}
\noindent 运行结果为:
\begin{figure}[htbp]
  \centering
    \includegraphics[scale=0.45]{4.8.png}
    \caption*{图9. \quad HHL算法的运行结果.}
\end{figure}

可见$n_l=10$时给出的解向量分量绝对值相当精确. 

不过需要注意的是,我们的程序无法判断分量的正负号. 这是因为我们实际测量的是$|b\rangle$寄存器的输出中分别得到$|0\rangle$和$|1\rangle$的概率,也就是$|x\rangle $分量模长的平方,从而无法直接判断$|0\rangle$和$|1\rangle$的相对相位. 可见根本原因仍在于HHL算法输出的解处于量子态,我们只能通过经典观测来了解它,这个过程中不可避免地损失了一些信息. 

想要确定$|x\rangle$的完整信息,我们可以在测量后对输出的量子态进行进一步的处理,比如引入多个不同可观测量$M_i$计算在解上的期望值$\langle x|M_i|x\rangle$等,但这会增加额外的计算复杂度和资源消耗.

\section*{五 \quad 总结与展望}
HHL算法在量子数值计算等方面有着广泛的应用价值, 但是目前仍然处在抽象的算法分析与描述阶段. 使用IBM qiskit量子计算等平台,可以对HHL量子算法的通用量子线路进行设计和仿真.  我们期待未来能有更多关于HHL算法理论优化或物理实现的研究,推动量子计算在实际问题求解中的应用.

\clearpage
\begin{thebibliography}{99}

\bibitem{1} Harrow, A. W., Hassidim, A., \& Lloyd, S. (2009). Quantum algorithm for solving linear systems of equations. Physical Review Letters, 103(15), 150502.
\bibitem{2} Dervovic, D., Herbster, M., Mountney, P., Severini, S., Usher, N., \& Wossnig, L. (2018). Quantum linear systems algorithms: a primer. arXiv preprint arXiv:1802.08227.
\bibitem{3} Mickael A. Nielsen, \& Isaac L. Chuang. (2010). Quantum Computation and Quantum Information: 10th Anniversary Edition. Cambridge University Press.
\bibitem{4}  Grover, L., \& Rudolph, T. (2002). Creating superpositions that correspond to effi
ciently integrable probability distributions. eprint: arXiv:quantph/0208112v1.
\bibitem{5} Qiskit Community, HHL Tutorial —Solving Linear Systems of Equations using Quantum Computing.Qiskit Textbook (GitHub), \quad  \nolinkurl{https://qiskit.org/textbook/ch-applications/hhl_tutorial.html}.
\end{thebibliography}

\end{document}
